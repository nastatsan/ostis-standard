\scsectionfamily{Часть 2 Стандарта OSTIS. Смысловое представление и онтологическая систематизация знаний в интеллектуальных компьютерных системах нового поколения}
\label{part_representation}

\scsection[\scneditors{Никифоров С.А.;Бобёр  Е.С.}\protect\scnmonographychapter{Глава 2.6. Языковые средства формального описания синтаксиса и денотационной семантики различных языков в интеллектуальных компьютерных системах нового поколения}]{Предметная область и онтология информационных конструкций и языков}
\label{intro_lang}
\input{Contents/part_kb/intro_lang.tex}

\scsubsection[\scnidtf{Предметная область и онтология языка внутреннего представления информационных конструкций в памяти ostis-систем}\protect\scnmonographychapter{Глава 2.1. Универсальный язык смыслового представления знаний и смысловое пространство}]{Предметная область и онтология внутреннего языка ostis-систем}
\label{intro_sc_code}
\input{Contents/part_kb/intro_lang/intro_sc_code.tex}

\scsubsection[\scnmonographychapter{Глава 2.2. Семейство внешних языков интеллектуальных компьютерных систем нового поколения, близких языку внутреннего смыслового представления знаний (SCg, SCs, SCn)}]{Предметная область и онтология внешних идентификаторов знаков, входящих в информационные конструкции внутреннего языка ostis-систем}
\label{intro_idtf}
\input{Contents/part_kb/intro_lang/intro_idtf}

\scsubsection[\scnmonographychapter{Глава 2.2. Семейство внешних языков интеллектуальных компьютерных систем нового поколения, близких языку внутреннего смыслового представления знаний (SCg, SCs, SCn)}]{Предметная область и онтология языка внешнего графического представления информационных конструкций внутреннего языка ostis-систем}
\label{intro_scg}
\input{Contents/part_kb/intro_lang/intro_scg.tex}

\scsubsubsection[\scnmonographychapter{Глава 2.2. Семейство внешних языков интеллектуальных компьютерных систем нового поколения, близких языку внутреннего смыслового представления знаний (SCg, SCs, SCn)}]{Предметная область и онтология синтаксиса языка внешнего графического представления информационных конструкций внутреннего языка ostis-систем}
\label{intro_scg_syntax}

\scsubsubsection[\scnmonographychapter{Глава 2.2. Семейство внешних языков интеллектуальных компьютерных систем нового поколения, близких языку внутреннего смыслового представления знаний (SCg, SCs, SCn)}]{Предметная область и онтология денотационной семантики языка внешнего графического представления информационных конструкций внутреннего языка ostis-систем}
\label{intro_scg_semantic}

\scsubsubsection[\scnmonographychapter{Глава 2.2. Семейство внешних языков интеллектуальных компьютерных систем нового поколения, близких языку внутреннего смыслового представления знаний (SCg, SCs, SCn)}]{Предметная область и онтология иерархического семейства подъязыков, семантически эквивалентных языку внешнего графического представления информационных конструкций внутреннего языка ostis-систем}
\label{intro_scg_sublang}

\scsubsection[\scnmonographychapter{Глава 2.2. Семейство внешних языков интеллектуальных компьютерных систем нового поколения, близких языку внутреннего смыслового представления знаний (SCg, SCs, SCn)}]{Предметная область и онтология языка внешнего линейного представления информационных конструкций внутреннего языка ostis-систем}
\label{intro_scs}
\input{Contents/part_kb/intro_lang/intro_scs.tex}

\scsubsubsection[\scnmonographychapter{Глава 2.2. Семейство внешних языков интеллектуальных компьютерных систем нового поколения, близких языку внутреннего смыслового представления знаний (SCg, SCs, SCn)}]{Предметная область и онтология синтаксиса языка внешнего линейного представления информационных конструкций внутреннего языка ostis-систем}
\label{intro_scs_syntax}

\scsubsubsection[\scnmonographychapter{Глава 2.2. Семейство внешних языков интеллектуальных компьютерных систем нового поколения, близких языку внутреннего смыслового представления знаний (SCg, SCs, SCn)}]{Предметная область и онтология денотационной семантики языка внешнего линейного представления информационных конструкций внутреннего языка ostis-систем}
\label{intro_scs_semantic}

\scsubsubsection[\scnmonographychapter{Глава 2.2. Семейство внешних языков интеллектуальных компьютерных систем нового поколения, близких языку внутреннего смыслового представления знаний (SCg, SCs, SCn)}]{Предметная область и онтология иерархического семейства подъязыков, семантически эквивалентных языку внешнего линейного представления информационных конструкций внутреннего языка ostis-систем}
\label{intro_scs_sublang}

\scsubsection[\scnmonographychapter{Глава 2.2. Семейство внешних языков интеллектуальных компьютерных систем нового поколения, близких языку внутреннего смыслового представления знаний (SCg, SCs, SCn)}]{Предметная область и онтология языка внешнего форматированного представления информационных конструкций внутреннего языка ostis-систем}
\label{intro_scn}
\input{Contents/part_kb/intro_lang/intro_scn.tex}

\scsubsubsection[\scnmonographychapter{Глава 2.2. Семейство внешних языков интеллектуальных компьютерных систем нового поколения, близких языку внутреннего смыслового представления знаний (SCg, SCs, SCn)}]{Предметная область и онтология синтаксиса языка внешнего форматированного представления информационных конструкций внутреннего языка ostis-систем}
\label{intro_scn_syntax}

\scsubsubsection[\scnmonographychapter{Глава 2.2. Семейство внешних языков интеллектуальных компьютерных систем нового поколения, близких языку внутреннего смыслового представления знаний (SCg, SCs, SCn)}]{Предметная область и онтология денотационной семантики языка внешнего форматированного представления информационных конструкций внутреннего языка ostis-систем}
\label{intro_scn_semantic}

\scsubsubsection[\scnmonographychapter{Глава 2.2. Семейство внешних языков интеллектуальных компьютерных систем нового поколения, близких языку внутреннего смыслового представления знаний (SCg, SCs, SCn)}]{Предметная область и онтология иерархического семейства подъязыков, семантически эквивалентных языку внешнего форматированного представления информационных конструкций внутреннего языка ostis-систем}
\label{intro_scn_sublang}

\scsection[\scneditor{Банцевич К.А.}\protect\scnmonographychapter{Глава 2.3. Структура баз знаний интеллектуальных компьютерных систем нового поколения: иерархическая система предметных областей и онтологий. Онтологии верхнего уровня. Формализация понятий семантической окрестности, предметной области и онтологии в интеллектуальных компьютерных системах нового поколения}]{Предметная область и онтология знаний и баз знаний ostis-систем}
\label{sd_knowledge}
\input{Contents/part_kb/sd_knowledge.tex}

\scsubsection[\scnidtf{Предметная область и онтология знаний о множествах}\protect\scnmonographychapter{Глава 2.4. Формальные онтологии базовых классов сущностей - множеств, связей, отношений, параметров, величин, чисел, структур, темпоральных сущностей}]{Предметная область и онтология множеств}
\label{sd_sets}
\input{Contents/part_kb/sd_sets.tex}

\scsubsection[\scnmonographychapter{Глава 2.4. Формальные онтологии базовых классов сущностей - множеств, связей, отношений, параметров, величин, чисел, структур, темпоральных сущностей}]{Предметная область и онтология связок и отношений}
\label{sd_rels}
\input{Contents/part_kb/sd_relations.tex}

\scsubsection[\scnmonographychapter{Глава 2.4. Формальные онтологии базовых классов сущностей - множеств, связей, отношений, параметров, величин, чисел, структур, темпоральных сущностей}]{Предметная область и онтология параметров, величин и шкал}
\label{sd_params}
\input{Contents/part_kb/sd_parameters_and_quantities.tex}

\scsubsection[\scnmonographychapter{Глава 2.4. Формальные онтологии базовых классов сущностей - множеств, связей, отношений, параметров, величин, чисел, структур, темпоральных сущностей}]{Предметная область и онтология чисел и числовых структур}
\input{Contents/part_kb/sd_numbers.tex}

\scsubsection[\scnmonographychapter{Глава 2.3. Структура баз знаний интеллектуальных компьютерных систем нового поколения: иерархическая система предметных областей и онтологий. Онтологии верхнего уровня. Формализация понятий семантической окрестности, предметной области и онтологии в интеллектуальных компьютерных системах нового поколения}]{Предметная область и онтология структур}
\label{sd_structures}
\input{Contents/part_kb/sd_structures.tex}

\scsubsection[\scnmonographychapter{Глава 2.4. Формальные онтологии базовых классов сущностей - множеств, связей, отношений, параметров, величин, чисел, структур, темпоральных сущностей}]{Предметная область и онтология темпоральных сущностей}
\label{sd_temp_entities}
\input{Contents/part_kb/sd_temp_entities.tex}

\scsubsubsection[\scnmonographychapter{Глава 2.4. Формальные онтологии базовых классов сущностей - множеств, связей, отношений, параметров, величин, чисел, структур, темпоральных сущностей}]{Предметная область и онтология ситуаций и событий, описывающих динамику баз знаний ostis-систем}
\label{sd_temp_know_base}
\input{Contents/part_kb/sd_temp_entities_kb.tex}

\scsubsection[\scnmonographychapter{Глава 2.4. Формальные онтологии базовых классов сущностей - множеств, связей, отношений, параметров, величин, чисел, структур, темпоральных сущностей}]{Предметная область и онтология пространственных сущностей различных форм}
\label{sd_spatial_entities}

\scsubsection[\scnmonographychapter{Глава 2.4. Формальные онтологии базовых классов сущностей - множеств, связей, отношений, параметров, величин, чисел, структур, темпоральных сущностей}]{Предметная область и онтология материальных сущностей}
\label{sd_material_entities}

\scsubsection[\scnmonographychapter{Глава 2.3. Структура баз знаний интеллектуальных компьютерных систем нового поколения: иерархическая система предметных областей и онтологий. Онтологии верхнего уровня. Формализация понятий семантической окрестности, предметной области и онтологии в интеллектуальных компьютерных системах нового поколения}]{Предметная область и онтология семантических окрестностей}
\label{sd_sem_neigh}
\input{Contents/part_kb/sd_semantic_neighborhood.tex}

\scsubsection[\scnmonographychapter{Глава 2.3. Структура баз знаний интеллектуальных компьютерных систем нового поколения: иерархическая система предметных областей и онтологий. Онтологии верхнего уровня. Формализация понятий семантической окрестности, предметной области и онтологии в интеллектуальных компьютерных системах нового поколения}]{Предметная область и онтология предметных областей}
\label{sd_sd}
\input{Contents/part_kb/sd_sd.tex}

\scsubsection[\scnmonographychapter{Глава 2.3. Структура баз знаний интеллектуальных компьютерных систем нового поколения: иерархическая система предметных областей и онтологий. Онтологии верхнего уровня. Формализация понятий семантической окрестности, предметной области и онтологии в интеллектуальных компьютерных системах нового поколения}]{Предметная область и онтология онтологий}
\label{sd_ontologies}
\input{Contents/part_kb/sd_ontologies.tex}

\scsubsection[\scneditors{Василевская А.П.;Зотов Н.В.;Орлов М.К.}\protect\scnmonographychapter{Глава 2.5. Смысловое представление логических формул и высказываний в различного вида логиках}]{Предметная область и онтология логических формул и высказываний}
\label{sd_logics}
\begin{SCn}

\scnsectionheader{\currentname}

\scnstartsubstruct

\scnheader{Предметная область логических формул, высказываний и формальных теорий}
\scniselement{предметная область}
\scnsdmainclasssingle{формальная теория}
\scnsdclass{высказывание;атомарное высказывание;неатомарное высказывание;фактографическое высказывание;логическая формула;атомарная логическая формула;неатомарная логическая формула;утверждение;определение;общезначимая логическая формула;противоречивая логическая формула;нейтральная логическая формула;выполнимая логическая формула;невыполнимая логическая формула;тавтология;квантор;формула существования;число значений переменной;кратность существования;единственное существование;логическая формула и единственность;открытая логическая формула;замкнутая логическая формула}
\scnsdrelation{предметная область\scnrolesign;аксиома\scnrolesign;теорема\scnrolesign;подформула*;логическая связка*;импликация*;если\scnrolesign;то\scnrolesign;эквиваленция*;конъюнкция*;дизъюнкция*;строгая дизъюнкция*;отрицание*;всеобщность*;неатомарное существование*;связываемые переменные\scnrolesign}

\scnheader{формальная теория}
\scnexplanation{\textbf{\textit{формальная теория}} — это множество высказываний, которые считаются истинными в рамках данной \textbf{\textit{формальной теории}}.}
\scnaddlevel{1}
\scnexplanation{Высказывания могут быть как фактографическими, так и логическими формулами. Некоторые высказывания считаются аксиомами, а другие доказываются на основе других высказываний в рамках этой же \textbf{\textit{формальной теории}}.}
\scnaddlevel{-1}
\scnexplanation{Каждая формальная теория интерпретируется (т.е. ее высказывания являются истинными) на какой-либо \textit{предметной области}, которая является максимальным из \textit{фактографических высказываний} (их \textit{объединением*}),  входящих в состав этой \textbf{\textit{формальной теории}}.}
\scnexplanation{Каждой \textbf{\textit{формальной теории}} соответствует одна \textit{предметная область}, которая входит в нее под атрибутом \textit{предметная область\scnrolesign}.}
\scnexplanation{Каждая \textbf{\textit{формальная теория}} может рассматриваться как конъюнктивное высказывание, априори истинное (с чьей-то точки зрения) при интерпретации на соответствующей \textit{предметной области}.}
\scnexplanation{Каждая \textbf{\textit{формальная теория}} задаётся алфавитом, формулами, аксиомами, правилами вывода.}
\scnaddlevel{1}
\scnrelfrom{источник}{\scncite{Serhievskaya2004}}
\scnaddlevel{-1}

\scnheader{предметная область\scnrolesign}
\scniselement{ролевое отношение}
\scnexplanation{\textbf{\textit{предметная область\scnrolesign}} -- это \textit{ролевое отношение}, связывающее \textit{формальную теорию} с \textit{предметной областью}, на которой данная \textit{формальная теория} интерпретируется (в рамках которой истинны \textit{высказывания}, входящие в состав этой \textit{формальной теории}).}
\scnexplanation{\textit{Предметная область} является максимальным фактографическим высказыванием \textit{формальной теории}, которая интерпретируется на данной \textit{предметной области}.}
\scnrelfrom{смотрите}{\nameref{sd_sd}}

\scnheader{аксиома\scnrolesign}
\scniselement{ролевое отношение}
\scnexplanation{\textbf{\textit{аксиома\scnrolesign}} -- это \textit{ролевое отношение}, связывающее \textit{формальную теорию} с \textit{высказыванием}, истинность которого не  требует доказательства в рамках этой \textit{формальной теории}.}

\scnheader{теорема\scnrolesign}
\scniselement{ролевое отношение}
\scnexplanation{\textbf{\textit{теорема\scnrolesign}} -- это \textit{ролевое отношение}, связывающее \textit{формальную теорию} с \textit{высказыванием}, истинность которого доказывается в рамках этой \textit{формальной теории}.}

\scnheader{высказывание}
\scnsubdividing{атомарное высказывание;неатомарное высказывание}
\scnsubdividing{фактографическое высказывание;логическая формула}
\scnexplanation{Под \textbf{\textit{высказыванием}} понимается некоторая \textit{структура} (в которую входят \textit{sc-константы} из некоторой предметной области и/или \textit{sc-переменные}) или \textit{логическая связка}, которая может трактоваться как истинная или ложная в рамках какой-либо \textit{предметной области}.}
\scnnote{Истинность \textbf{\textit{высказывания}} задается путем указания принадлежности знака этого высказывания \textit{формальной теории}, соответствующей данной \textit{предметной области}. Ложность высказывания задается путем указания принадлежности знака \textit{отрицания*} этого высказывания данной \textit{формальной теории}.}
\scnaddlevel{1}
\scnnote{Явно указанная непринадлежность \textbf{\textit{высказывания}} \textit{формальной теории} может говорить как о его ложности в рамках данной теории (если это указано рассмотренным выше образом), так и о том, что данное  \textbf{\textit{высказывание}} вообще не рассматривается в данной \textit{формальной теории} (например, использует понятия, не принадлежащие данной \textit{предметной области}).}
\scnnote{Одно и то же \textbf{\textit{высказывание}} может быть истинно в рамках одной \textit{формальной теории} и ложно в рамках другой.}
\scnaddlevel{-1}

\scnheader{высказывание формальной теории\scnrolesign}
\scniselement{неосновное понятие}
\scnsubdividing{истинное высказывание\scnrolesign\\
	\scnaddlevel{1}
		\scnidtf{высказывание, истинное в рамках данной формальной теории\scnrolesign}
		\scnidtf{высказывание, знак которого принадлежит данной формальной теории\scnrolesign}
	\scnaddlevel{-1}
	;ложное высказывание\scnrolesign\\
	\scnaddlevel{1}
		\scnidtf{высказывание, ложное в рамках данной формальной теории\scnrolesign}
		\scnidtf{высказывание, знак отрицания которого принадлежит данной формальной теории\scnrolesign}
	\scnaddlevel{-1}
	;нечеткое высказывание\scnrolesign\\
	\scnaddlevel{1}
		\scnidtf{гипотетическое высказывание\scnrolesign}
		\scnidtf{высказывание, возможно истинное или ложное в рамках данной формальной теории\scnrolesign}
		\scnidtf{высказывание, истинное или ложное в рамках данной формальной теории с некоторой вероятностью\scnrolesign}
	\scnaddlevel{-1}
	;бессмысленное высказывание\scnrolesign\\
	\scnaddlevel{1}
		\scnidtf{высказывание, бессмысленное в рамках данной формальной теории\scnrolesign}
		\scnidtf{высказывание, не рассматриваемое в рамках данной формальной теории\scnrolesign}
		\scnexplanation{Высказывание является бессмысленным в рамках заданной формальной теории, если в какое-либо \textit{атомарное высказывание} в его составе (или в само это высказывание, если оно является атомарным) входит какая-либо \textit{sc-константа}, не являющаяся элементом предметной области, описываемой указанной \textit{формальной теорией}.}
	\scnaddlevel{-1}}

\scnheader{атомарное высказывание}
\scnsubset{структура}
\scnsubdividing{атомарное фактографическое высказывание;атомарная логическая формула}
\scndefinition{\textbf{\textit{атомарное высказывание}} -- это \textit{высказывание}, которое содержит хотя бы один \textit{sc-элемент}, не являющийся знаком другого \textit{высказывания}.}
\scnheader{неатомарное высказывание}
\scndefinition{\textbf{\textit{неатомарное высказывание}} -- это \textit{высказывание}, в состав которого входят только знаки других \textit{высказываний}.}
\scnnote{Следует отметить, что мы не можем говорить об истинности либо ложности \textbf{\textit{неатомарного высказывания}} в рамках какой-либо \textit{формальной теории}, в случае, когда невозможно установить истинность либо ложность любого из его элементов в рамках этой же \textit{формальной теории}.}

\scnheader{фактографическое высказывание}
\scnsuperset{атомарное фактографическое высказывание}
\scnexplanation{Под \textit{фактографическим высказыванием} понимается:
\begin{scnitemize}
    \item \textit{атомарное высказывание}, в состав которого не входит ни одна \textit{sc-переменная};
    \item \textit{неатомарное высказывание}, все элементы которого также являются \textbf{\textit{фактографическими высказываниями}}.
\end{scnitemize}
}

\scnheader{логическая формула}
\scnexplanation{Под \textit{логической формулой} понимается:
\begin{scnitemize}
    \item \textit{атомарное высказывание}, в состав которого входит хотя бы одна \textit{sc-переменная};
    \item \textit{неатомарное высказывание}, хотя бы один элемент которого является \textbf{\textit{логической формулой}}.
\end{scnitemize}}
\scnsubdividing{атомарная логическая формула;неатомарная логическая формула}
\scnsubdividing{открытая логическая формула;замкнутая логическая формула}

\scnheader{атомарная логическая формула}
\scnidtf{обобщенная структура}
\scnidtf{атомарная формула существования}
\scnexplanation{Под \textbf{\textit{атомарной логической формулой}} понимается \textit{атомарное высказывание}, которое является \textit{логической формулой}.}
\scnexplanation{\textbf{\textit{Атомарная логическая формула}} -- это  логическая формула, которая не содержит логических связок.}
\scnnote{По умолчанию \textbf{\textit{атомарная логическая формула}} трактуется как \textit{высказывание} о существовании, то есть наличия в памяти значений, соответствующих всем \textit{sc-переменным}, входящим в состав данной формулы и не попадающих под действие какого-либо другого \textit{квантора} (указанного явно или по умолчанию). Таким образом, на все \textit{sc-переменные}, входящие в состав \textbf{\textit{атомарной логической формулы}} и не попадающие под действие какого-либо другого \textit{квантора}, неявно накладывается квантор \textit{существования*}.}
\scnaddlevel{1}
\scnrelfrom{основной sc-идентификатор}{\scnfilelong{Примечание про высказывание о существовании}}
\scnaddlevel{-1}

\scnheader{неатомарная логическая формула}
\scnsubdividing{общезначимая логическая формула;противоречивая логическая формула;нейтральная логическая формула}
\scnsubdividing{выполнимая логическая формула;невыполнимая логическая формула}
\scnsuperset{тавтология}
\scnexplanation{Под \textbf{\textit{неатомарной логической формулой}} понимается \textit{неатомарное высказывание}, которое является \textit{логической формулой}.}
\scnnote{Для того, чтобы рассмотреть типологию \textbf{\textit{неатомарных логических формул}}, будем говорить, что исследуется истинность самой \textbf{\textit{неатомарной логической формулы}} и всех ее \textit{подформул*} в рамках одной и той же \textit{формальной теории}, при этом не важно, какой именно. Также считается, что в рассматриваемой \textit{формальной теории} каждая \textit{подформула*} рассматриваемой \textbf{\textit{неатомарной логической формулы}} в рамках этой \textit{формальной теории} может однозначно трактоваться как либо истинная, либо ложная. В противном случае мы не можем говорить об истинности либо ложности исходной \textbf{\textit{неатомарной логической формулы}} в рамках этой \textit{формальной теории}.}
\scnrelfrom{описание примера}{Примеры неатомарных логических формул}

\scnheader{подформула*}
\scnidtf{частная формула*}
\scniselement{бинарное отношение}
\scniselement{ориентированное отношение}
\scniselement{транзитивное отношение}
\scndefinition{Будем называть \textbf{\textit{подформулой*}} \textit{неатомарной логической формулы} \textbf{\textit{fi}} любую \textit{логическую формулу} \textbf{\textit{fj}}, являющуюся элементом исходной формулы \textbf{\textit{fi}}, а также любую \textbf{\textit{подформулу*}} формулы \textbf{\textit{fj}}.}
\scnrelfrom{описание примера}{
\scnfilescg{figures/sd_logical_formulas/subformula.png}}
\scnaddlevel{1}
\scniselement{sc.g-текст}
\scnaddlevel{-1}

\scnheader{утверждение}
\scnidtf{текст логической формулы}
\scndefinition{\textbf{\textit{утверждение}} -- это \textit{семантическая окрестность} некоторой \textit{логической формулы}, в которую входит полный текст этой \textit{логической формулы}, а также факт принадлежности этой \textit{логической формулы} некоторой \textit{формальной теории}.}
\scnexplanation{Знак \textit{логической формулы}, семантическая окрестность которой представляет собой утверждение, является \textit{главным ключевым sc-элементом\scnrolesign} в рамках этого \textbf{\textit{утверждения}}. Знаки понятий соответствующей \textit{предметной области}, которые входят в состав какой-либо \textit{подформулы*} указанной \textit{логической формулы}, будут \textit{ключевыми sc-элементами\scnrolesign} в рамках этого \textbf{\textit{утверждения}}.

Полный текст некоторой \textit{логической формулы} включает в себя:
\begin{scnitemize}
    \item знак самой этой \textit{логической формулы};
    \item знаки всех ее \textit{подформул*};
    \item элементы всех \textit{логических формул}, знаки которых попали в данную структуру;
    \item все пары принадлежности, связывающие \textit{логические формулы}, знаки которых попали в данную структуру, с их компонентами.
\end{scnitemize}
Таким образом, факт принадлежности (истинности) логической формулы нескольким \textit{формальным теориям} будет порождать новое утверждение для каждой такой \textit{формальной теории}. Текст \textbf{\textit{утверждения}} входит в состав \textit{логической онтологии}, соответствующей \textit{предметной области}, на которой интерпретируется \textit{главный ключевой sc-элемент\scnrolesign} данного утверждения.}
\scntext{правило идентификации экземпляров}{\textbf{\textit{утверждения}} в рамках \textit{Русского языка} именуются по следующим правилам:
\begin{scnitemize}
    \item в начале идентификатора пишется сокращение \textbf{Утв.};
    \item далее в круглых скобках через точку с запятой перечисляются основные идентификаторы \textit{ключевых \mbox{sc-элементов}\scnrolesign} данного \textbf{\textit{утверждения}}. Порядок определяется в каждом конкретном случае в зависимости от того, свойства каких из этих \textit{понятий} описывает данное \textbf{\textit{утверждение}} в большей или меньшей степени.
\end{scnitemize}
}
\scnaddlevel{1}
\scntext{описание примера}{\textit{Утв. (параллельность*; секущая*)}}
\scnnote{Могут быть исключения для \textbf{\textit{утверждений}}, названия которых закрепились исторически, например, \textit{Теорема Пифагора}, \textit{Аксиома о прямой и точке}.}
\scnaddlevel{-1}
\scnrelfrom{описание примера}{\scnfilescg{figures/sd_logical_formulas/statement.png}}
\scnaddlevel{1}
\scnnote{Утверждение показывает, что соответствующие углы при пересечении параллельных прямых секущей равны.}
\scniselement{sc.g-текст}
\scnaddlevel{-1}

\scnheader{определение}
\scnidtf{текст определения}
\scnsubset{утверждение}
\scndefinition{\textbf{\textit{определение}} -- это \textit{утверждение}, \textit{главным ключевым sc-элементом\scnrolesign} которого является связка \textit{эквиваленции*}, однозначно определяющая некоторое понятие на основе других понятий.}
\scnnote{Каждое определение имеет ровно один \textit{ключевой sc-элемент\scnrolesign} (не считая \textit{главного ключевого sc-элемента\scnrolesign}).}
\scnnote{Для одного и того же понятия в рамках одной \textit{формальной теории} может существовать несколько \textit{утверждений об эквиваленции*}, однозначно задающих некоторое понятие на основе других, однако только одно такое \textit{утверждение} в рамках этой \textit{формальной теории} может быть отмечено как \textbf{\textit{определение}}. Остальные \textit{утверждения об эквиваленции*} могут трактоваться как \textit{пояснения} данного понятия.}
\scntext{правило идентификации экземпляров}{\textbf{\textit{определения}} в рамках \textit{Русского языка} именуются по следующим правилам:
\begin{scnitemize}
    \item в начале идентификатора пишется сокращение \textbf{Опр.};
    \item далее в круглых скобках через точку с запятой записывается основной идентификатор  \textit{ключевого sc-элемента\scnrolesign} данного \textbf{\textit{определения}}.
\end{scnitemize}
}
\scnaddlevel{1}
\scntext{описание примера}{\textit{Опр. (ромб)}}
\scnaddlevel{-1}
\scnrelfrom{описание примера}{\scnfilescg{figures/sd_logical_formulas/definition.png}}
\scnaddlevel{1}
\scnnote{Определение показывает, что ромб — это четырёхугольник, у которого все стороны равны.}
\scniselement{sc.g-текст}
\scnaddlevel{-1}

\scnheader{общезначимая логическая формула}
\scnidtf{тождественно истинная логическая формула}
\scnsubset{выполнимая логическая формула}
\scnsubset{тавтология}
\scndefinition{\textbf{\textit{общезначимая логическая формула}} -- это \textit{логическая формула}, для которой не существует \textit{формальной теории}, в рамках которой она была бы ложной с учетом истинности и ложности всех ее \textit{подформул*} в рамках этой же \textit{формальной теории}.}
\scnrelfrom{описание примера}{
\scnfilescg{figures/sd_logical_formulas/valid_formula.png}}
\scnaddlevel{1}
\scnrelfrom{основной sc-идентификатор}{\scnfilelong{закон тождества}}
\scniselement{sc.g-текст}
\scnaddlevel{-1}

\scnheader{противоречивая логическая формула}
\scnidtf{тождественно ложная логическая формула}
\scnsubset{невыполнимая логическая формула}
\scnsubset{тавтология}
\scndefinition{\textbf{\textit{противоречивая логическая формула}} -- это \textit{логическая формула}, для которой не существует \textit{формальной теории}, в рамках которой она была бы истинной с учетом истинности и ложности всех ее \textit{подформул*} в рамках этой же \textit{формальной теории}.}
\scnrelfrom{описание примера}{
\scnfilescg{figures/sd_logical_formulas/contradiction_formula.png}}
\scnaddlevel{1}
\scnrelfrom{основной sc-идентификатор}{\scnfilelong{закон противоречия}}
\scniselement{sc.g-текст}
\scnaddlevel{-1}

\scnheader{нейтральная логическая формула}
\scnsubset{выполнимая логическая формула}
\scndefinition{\textbf{\textit{нейтральная логическая формула}} -- это \textit{логическая формула}, для которой существует хотя бы одна \textit{формальная теория}, в рамках которой эта формула ложна, и хотя бы одна \textit{формальная теория}, в рамках которой эта формула истинна.}
\scnrelfrom{описание примера}{
\scnfilescg{figures/sd_logical_formulas/neutral_formula.png}}
\scnaddlevel{1}
\scnnote{В евклидовой геометрии в плоскости через точку, не лежащую на данной прямой, можно провести одну и только одну прямую, параллельную данной. В геометрии Лобачевского данный постулат является ложным.}
\scnnote{В сферической геометрии все прямые пересекаются.}
\scniselement{sc.g-текст}
\scnaddlevel{-1}

\scnheader{непротиворечивая логическая формула}
\scnidtf{выполнимая логическая формула}
\scndefinition{\textbf{\textit{непротиворечивая логическая формула}} -- это \textit{логическая формула}, для которой существует хотя бы одна \textit{формальная теория}, в рамках которой эта формула истинна.}
\scnreltoset{объединение}{нейтральная логическая формула;общезначимая логическая формула}

\scnheader{необщезначимая логическая формула}
\scnidtf{невыполнимая логическая формула}
\scndefinition{\textbf{\textit{необщезначимая логическая формула}} -- это \textit{логическая формула}, для которой существует хотя бы одна \textit{формальная теория}, в рамках которой эта формула ложна.}
\scnreltoset{объединение}{нейтральная логическая формула;противоречивая логическая формула}
 
\scnheader{тавтология}
\scndefinition{\textbf{\textit{тавтология}} -- это \textit{логическая формула}, которая является либо только истинной, либо только ложной в рамках всех \textit{формальных теорий}, в которых можно установить ее истинность или ложность.}
\scnexplanation{\textbf{\textit{тавтология}} -- это такая \textit{логическая формула}, которая является либо \textit{общезначимой логической формулой}, либо \textit{противоречивой логической формулой}.}

\scnheader{логическая связка*}
\scnidtf{неатомарная логическая формула}
\scnidtf{логический оператор*}
\scnidtf{пропозициональная связка*}
\scniselement{класс связок разной мощности}
\scnrelto{семейство подмножеств}{неатомарное высказывание}
\scndefinition{\textbf{\textit{логическая связка*}} -- это отношение (класс связок), связками которого являются \textit{высказывания}.}
\scnexplanation{\textbf{\textit{логическая связка*}} -- это \textit{отношение}, областью определения которого является множество \textit{высказываний}, при этом само это отношение и некоторые его подмножества могут быть \textit{классами связок разной мощности}.}

\scnheader{конъюнкция*}
\scnidtf{логическое и*}
\scnidtf{логическое умножение*}
\scnsubset{логическая связка*}
\scniselement{неориентированное отношение}
\scniselement{класс связок разной мощности}
\scndefinition{\textbf{\textit{конъюнкция*}} -- это множество конъюнктивных \textit{высказываний}, каждое из которых истинно в рамках некоторой \textit{формальной теории} только в том случае, когда все его компоненты истинны в рамках этой же \textit{формальной теории}.}
\scnnote{\textbf{\textit{конъюнкция*}} атомарных формул может быть заменена на атомарную формулу, полученную путём объединения исходных атомарных формул.}
\scnaddlevel{1}
\scnrelfrom{описание примера}{
\scnfilescg{figures/sd_logical_formulas/conjunction_triangles.png}}
\scnaddlevel{1}
\scnexplanation{Данные конструкции эквивалентны по принципу $\exists x T(x) \land \exists x PT(x) \ \Longrightarrow \ \exists x (T(x) \land PT(x))$}
\scnaddlevel{1}
\scnexplanation{Следует помнить про \textbf{\textit{Примечание про высказывание о существовании}}.}
\scnaddlevel{-1}
\scniselement{sc.g-текст}
\scnaddlevel{-1}
\scnaddlevel{-1}
\scnrelfrom{описание примера}{
\scnfilescg{figures/sd_logical_formulas/conjunction.png}}
\scnaddlevel{1}
\scniselement{sc.g-текст}
\scnaddlevel{-1}

\scnheader{дизъюнкция*}
\scnidtf{логическое или*}
\scnidtf{логическое сложение*}
\scnidtf{включающее или*}
\scnsubset{логическая связка*}
\scniselement{неориентированное отношение}
\scniselement{класс связок разной мощности}
\scndefinition{\textbf{\textit{дизъюнкция*}} -- это множество дизъюнктивных \textit{высказываний}, каждое из которых истинно в рамках некоторой \textit{формальной теории} только в том случае, когда хотя бы один его компонент является истинным в рамках этой же \textit{формальной теории}.}
\scnrelfrom{описание примера}{
\scnfilescg{figures/sd_logical_formulas/disjunction.png}}
\scnaddlevel{1}
\scniselement{sc.g-текст}
\scnaddlevel{-1}

\scnheader{отрицание*}
\scnsubset{логическая связка*}
\scnsubset{синглетон}
\scndefinition{\textbf{\textit{отрицание*}} -- это множество \textit{высказываний} об отрицании, каждое из которых истинно в рамках некоторой \textit{формальной теории} только в том случае, когда его единственный элемент является ложным в рамках этой же \textit{формальной теории}.}
\scnrelfrom{описание примера}{
\scnfilescg{figures/sd_logical_formulas/negation.png}}
\scnaddlevel{1}
\scniselement{sc.g-текст}
\scnaddlevel{-1}

\scnheader{строгая дизъюнкция*}
\scnidtf{сложение по модулю 2*}
\scnidtf{исключающее или*}
\scnidtf{альтернатива*}
\scnsubset{логическая связка*}
\scniselement{неориентированное отношение}
\scniselement{класс связок разной мощности}
\scndefinition{\textbf{\textit{строгая дизъюнкция*}} -- это множество строго дизъюнктивных \textit{высказываний}, каждое из которых истинно в рамках некоторой \textit{формальной теории} только в том случае, когда ровно один его компонент является истинным в рамках этой же \textit{формальной теории}.}
\scnrelfrom{описание примера}{
\scnfilescg{figures/sd_logical_formulas/strictDisjunction.png}}
\scnaddlevel{1}
\scniselement{sc.g-текст}
\scnaddlevel{-1}
\scnrelfrom{описание примера}{
\scnfilescg{figures/sd_logical_formulas/strict_disjunction_triangle.png}}
\scnaddlevel{1}
\scnexplanation{Данная неатомарная логическая формула содержит следующую информацию: для любых переменных \_triangle если \_triangle является треугольником, то \_triangle является или тупоугольным треугольником, или остроугольным треугольником, или прямоугольным треугольником.}
\scniselement{sc.g-текст}
\scnaddlevel{-1}
\scnnote{\textbf{\textit{строгая дизъюнкция*}} может быть представлена как \textit{дизъюнкция} \textit{конъюнкции} \textit{отрицания} первой логической формулы и второй логической формулы и \textit{конъюнкции} первой логической формулы и \textit{отрицания} второй логической формулы. Также она может быть представлена и ввиде \textit{конъюнкции} \textit{дизъюнкций} двух логических формул и их \textit{отрицаний}.}
\scnaddlevel{1}
\scnrelfrom{описание примера}{	\scnfilescg{figures/sd_logical_formulas/strict_disjunction_representation.png}}
\scnaddlevel{1}
\scniselement{sc.g-текст}
\scnaddlevel{-1}
\scnaddlevel{-1}

\scnheader{импликация*}
\scnidtf{логическое следование*}
\scnsubset{логическая связка*}
\scniselement{бинарное отношение}
\scniselement{ориентированное отношение}
\scndefinition{\textbf{\textit{импликация*}} -- это множество импликативных \textit{неатомарных высказываний}, каждое из которых состоит из посылки (первый компонент \textit{высказывания}) и следствия (второй компонент \textit{высказывания}).}
\scnnote{Каждое импликативное \textit{высказывание} ложно в рамках некоторой \textit{формальной теории} в том случае, когда его посылка истинна, а заключение ложно в рамках этой же \textit{формальной теории}. В других случаях такое \textit{высказывание} истинно.}
\scnnote{По умолчанию на все переменные, входящие в обе части высказывания об \textbf{\textit{имликации*}} (или хотя бы одну из \textit{подформул*} каждой части) неявно накладывается квантор \textit{всеобщности*}, при условии, что эти переменные не связаны другим \textit{квантором}, указанным явно.}
\scnrelfrom{описание примера}{
\scnfilescg{figures/sd_logical_formulas/implication.png}}
\scnaddlevel{1}
\scniselement{sc.g-текст}
\scnaddlevel{-1}
\scnnote{\textbf{\textit{импликация*}} может быть представлена как \textit{дизъюнкция} \textit{отрицания} первой логической формулы и второй логической формулы или же как \textit{отрицание} \textit{конъюнкции} первой логической формулы и \textit{отрицания} второй логической формулы.}
\scnaddlevel{1}
\scnrelfrom{описание примера}{
\scnfilescg{figures/sd_logical_formulas/implication_representation.png}}
\scnaddlevel{1}
\scniselement{sc.g-текст}
\scnaddlevel{-1}
\scnaddlevel{-1}
\scnrelfrom{описание примера}{
\scnfilescg{figures/sd_logical_formulas/implication_triangle.png}}
\scnaddlevel{1}
\scnexplanation{Данная неатомарная логическая формула содержит следующую информацию: для любых переменных \_triangle и \_angle если \_triangle является прямоугольным треугольником, то синус его внутреннего угла \_angle равен единице.}
\scniselement{sc.g-текст}
\scnaddlevel{-1}

\scnheader{если\scnrolesign}
\scnidtf{посылка\scnrolesign}
\scnsubset{1\scnrolesign}
\scniselement{ролевое отношение}
\scndefinition{\textbf{\textit{если\scnrolesign}} -- это \textit{ролевое отношение}, используемое в связках \textit{импликации*} для указания посылки.}

\scnheader{то\scnrolesign}
\scnidtf{следствие\scnrolesign}
\scnsubset{2\scnrolesign}
\scniselement{ролевое отношение}
\scndefinition{\textbf{\textit{то\scnrolesign}} -- это \textit{ролевое отношение}, используемое в связках \textit{импликации*} для указания следствия.}

\scnheader{эквиваленция*}
\scnidtf{эквивалентность*}
\scnsubset{логическая связка*}
\scniselement{бинарное отношение}
\scniselement{неориентированное отношение}
\scndefinition{\textbf{\textit{эквиваленция*}} -- это множество \textit{высказываний} об эквивалентности, каждое из которых истинно в рамках некоторой \textit{формальной теории} только в тех случаях, когда оба его компонента одновременно либо истинны в рамках этой же \textit{формальной теории}, либо ложны.}
\scnnote{По умолчанию на все переменные, входящие в обе части высказывания об \textbf{\textit{эквиваленции*}} (или хотя бы одну из \textit{подформул*} каждой части) неявно накладывается квантор \textit{всеобщности*}, при условии, что эти переменные не связаны другим \textit{квантором}, указанным явно.}
\scnrelfrom{описание примера}{
\scnfilescg{figures/sd_logical_formulas/equivalent.png}}
\scnaddlevel{1}
\scniselement{sc.g-текст}
\scnaddlevel{-1}
\scnnote{\textbf{\textit{эквиваленция*}} двух логических формул может быть представлена как \textit{дизъюнкция} \textit{конъюнкции} этих двух логическх формул и \textit{конъюнкции} \textit{отрицаний} этих двух логических формул.}
\scnaddlevel{1}
\scnrelfrom{описание примера}{
\scnfilescg{figures/sd_logical_formulas/equivalence_representation.png}}
\scnaddlevel{1}
\scniselement{sc.g-текст} 
\scnaddlevel{-1}
\scnaddlevel{-1}

\scnheader{квантор}
\scnsubset{логическая связка*}
\scndefinition{\textbf{\textit{квантор}} — это \textit{отношение}, каждая связка которой задает истинность множества \textit{логических формул}, входящих в ее состав, при выполнении дополнительных условий, связанных с некоторыми из переменных, входящих в состав этих \textit{логических формул}.}
\scnnote{Будем говорить, что переменные связаны \textbf{\textit{квантором}} или попадают под область действия данного \textbf{\textit{квантора}} (имея в виду конкретную связку конкретного \textbf{\textit{квантора}}).}
\scnnote{В состав каждой связки каждого \textbf{\textit{квантора}} входит \textit{атомарная формула}, являющаяся \textit{тривиальной структурой}, в которой перечислены переменные, связанные данным \textbf{\textit{квантором}}.}

\scnheader{всеобщность*}
\scnidtf{квантор всеобщности*}
\scnidtf{квантор общности*}
\scniselement{квантор}
\scniselement{ориентированное отношение}
\scniselement{класс связок разной мощности}
\scndefinition{\textbf{\textit{всеобщность}} -- это \textit{квантор}, для каждой связки которого, истинной в рамках некоторой \textit{формальной теории}, выполняется следующее утверждение: все формулы, входящие в состав этой связки истинны в рамках этой же \textit{формальной теории} при всех (любых) возможных значениях всех элементов множества \textit{связываемых переменных\scnrolesign} входящего в эту связку.}
\scnnote{Каждая связка \textit{квантора} \textbf{\textit{всеобщность*}} может быть представлена как \textit{конъюнкция*} (потенциально бесконечная) исходных \textit{логических формул}, входящих в состав этой связки, в каждой из которых все \textit{связанные переменные\scnrolesign} заменены на их возможные значения.}
\scnnote{Квантор \textbf{\textit{всеобщности*}} зачастую обозначается "$\forall$" \ и читается как "для всех"{}, "для каждого"{}, "для любого"{} или "все"{}, "каждый"{}, "любой".}
\scnrelfrom{описание примера}{
\scnfilescg{figures/sd_logical_formulas/universality.png}}
\scnaddlevel{1}
\scniselement{sc.g-текст}
\scnaddlevel{-1}

\scnheader{формула существования}
\scnidtf{существование*}
\scnsubdividing{атомарная логическая формула;неатомарное существование*}

\scnheader{неатомарное существование*}
\scnidtf{квантор неатомарного существования*}
\scniselement{квантор}
\scniselement{ориентированное отношение}
\scniselement{класс связок разной мощности}
\scndefinition{\textbf{\textit{неатомарное существование*}} -- это \textit{квантор}, для каждой связки которого, истинной в рамках некоторой \textit{формальной теории}, выполняется следующее утверждение: существуют значения всех элементов множества \textit{связываемых переменных\scnrolesign} входящего в эту связку, такие, что все формулы, входящие в состав этой связки истинны в рамках этой же \textit{формальной теории}.}
\scnnote{Каждая связка \textit{квантора} \textbf{\textit{неатомарное существование*}} может быть представлена как \textit{дизъюнкция*} (потенциально бесконечная) исходных \textit{логических формул}, входящих в состав этой связки, в каждой из которых все \textit{связанные переменные\scnrolesign} заменены на их возможные значения.}
\scnnote{квантор \textbf{\textit{существования*}} зачастую обозначается "$\exists$" \ и читается как "существует"{}, "для некоторого"{}, "найдется".}
\scnrelfrom{описание примера}{
\scnfilescg{figures/sd_logical_formulas/non_atomicExistence.png}}
\scnaddlevel{1}
\scniselement{sc.g-текст}
\scnaddlevel{-1}

\scnheader{число значений переменной}
\scniselement{параметр}
\scnexplanation{Каждый элемент \textit{параметра} \textbf{\textit{число значений переменной}} представляет собой класс ориентированных пар, первым компонентом которых является знак \textit{логической формулы}, вторым -- \textit{sc-переменная}, имеющая в рамках данной \textit{логической формулы} ограниченное известное число значений, при которых данная формула является истинной в рамках соответствующей \textit{формальной теории}.}
\scnnote{Отметим, что в случае \textit{атомарной логической формулы} каждая такая связка связывает знак формулы и знак принадлежащей ей \textit{sc-переменной}, т.е. является, по сути, частным случаем пары принадлежности. В случае \textit{неатомарной логической формулы} указанная \textit{sc-переменная} может принадлежать любой из \textit{подформул*} исходной формулы.}
\scnnote{\textit{измерением*} значения параметра \textbf{\textit{число значений переменной}} является некоторое \textit{число}, задающее количество значений \textit{sc-переменных} в рамках \textit{логической формулы}.}

\scnheader{кратность существования}
\scniselement{параметр}
\scnrelfrom{область определения параметра}{формула существования}
\scnhaselement{единственное существование}
\scnexplanation{Каждый элемент \textit{параметра} \textbf{\textit{кратность существования}} представляет собой класс логических \textit{формул существования}, для которых  при интерпретации на соответствующей \textit{предметной области} существует ограниченное общее для всех таких формул число комбинаций значений переменных, при которых указанные формулы являются истинными в рамках соответствующей \textit{формальной теории}.}
\scnaddlevel{1}
\scnnote{\textit{измерением*} каждого значения \textbf{\textit{кратности существования}} является некоторое \textit{число}, задающее количество таких комбинаций.}
\scnaddlevel{-1}

\scnheader{единственное существование}
\scnidtf{однократное существование}
\scnidtf{формула существования и единственности}
\scnnote{\textbf{\textit{единственное существование}} зачастую обозначается "$\exists!$" \ и читается как "существует и единственный".}

\scnheader{логическая формула и единственность}
\scnsubset{логическая формула}
\scnsubset{единственное существование}
\scnexplanation{Каждый элемент множества \textbf{\textit{логическая формула и единственность}} представляет собой \textit{логическую формулу} (\textit{атомарную} или \textit{неатомарную}), для которой дополнительно уточняется, что при ее интерпретации на некоторой предметной области существует только один набор значений переменных, входящих в эту формулу (или ее \textit{подформулы*}), при котором указанная логическая формула истинна в рамках \textit{формальной теории}, в которую входит данная \textit{предметная область}.}
\scnrelfrom{описание примера}{
\scnfilescg{figures/sd_logical_formulas/unique_existance.png}}
\scnaddlevel{1}
\scnnote{Данная формула показывает, что в рамках формальной теории геометрии Евклида существует только один прямоугольный треугольник с некоторым периметром, являющийся равнобедренным.}

\scniselement{sc.g-текст}
\scnaddlevel{-1}

\scnheader{связываемые переменные\scnrolesign}
\scniselement{ролевое отношение}
\scndefinition{\textbf{\textit{связываемые переменные\scnrolesign}} -- это \textit{ролевое отношение}, которое связывает связку конкретного \textit{квантора} с множеством переменных, которые связаны этим квантором.}

\scnheader{открытая логическая формула}
\scndefinition{\textbf{\textit{открытая логическая формула}} -- это \textit{логическая формула}, в рамках которой (и всех ее \textit{подформул*}) существует хотя бы одна переменная, не связанная никаким \textit{квантором}.}

\scnheader{замкнутая логическая формула}
\scndefinition{\textbf{\textit{замкнутая логическая формула}} -- это \textit{логическая формула}, в рамках которой (и всех ее \textit{подформул*}) не существует переменных, не связанных каким-либо \textit{квантором}.}

\scnheader{нечеткая логическая формула}
\scndefinition{любая нечеткая высказывательная переменная или константа из замкнутого интервала между 0 и 1.}
\scndefinition{выражение, полученное из нечетких логических формул применением к ним любого конечного числа логических операций.}

\scnheader{предикат}
\scndefinition{функция, характеризующая некоторое отношение, и областью значений которой являются 0 или 1.}

\scnheader{нечеткий предикат}
\scndefinition{функция, которая определена на каком-либо множестве и областью значений которой является замкнутый интервал между 0 и 1.}

\scnheader{Примеры неатомарных логических формул}
\scneqtoset{\scgfileitem{figures/sd_logical_formulas/example_line_segment_sum.png}\\
\scnaddlevel{1}
\scnrelfrom{описание примера}{
\scnfilescg{figures/sd_logical_formulas/example_line_segment_sum_note.png}}
\scnexplanation{AB+BC=AC}
\scniselement{sc.g-текст}
\scnaddlevel{-1}
;
\scgfileitem{figures/sd_logical_formulas/example_line_segment_diff.png}\\
\scnaddlevel{1}
\scnrelfrom{описание примера}{
\scnfilescg{figures/sd_logical_formulas/example_line_segment_diff_note.png}}
\scnexplanation{AB-AC=CB}
\scniselement{sc.g-текст}
\scnaddlevel{-1}
}

\bigskip
\scnendstruct \scnendcurrentsectioncomment

\end{SCn}

\scsubsection[\scneditors{Василевская А.П.;Орлов М.К.}\protect\scnmonographychapter{Глава 2.5. Смысловое представление логических формул и высказываний в различного вида логиках}]{Предметная область и онтология логических sc-языков}
\label{sd_logics}
\begin{SCn}

    \scnsectionheader{\currentname}
    
    \scnstartsubstruct
    
    \scnheader{Предметная область логических языков}
    \scniselement{предметная область}
    \scnsdmainclasssingle{логический язык}
    %\scnsdclass{логический язык}
    \scnsdrelation{логический язык*}
    
    \scnheader{логический язык}
    \scnidtf{формальный язык}
    \scnsubset{язык}
    \scndefinition{искусственный язык логики, предназначенный для воспроизведения логических форм контекстов естественного языка, а также выражения логических законов и способов правильных рассуждений в логических теориях, строящихся в данном языке.}
    
    \scnheader{логический язык*}
    \scnidtf{быть языком логики*} 
    \scnidtf{быть языком логической модели*}
    
    \bigskip
    \scnendstruct
    
    \scnstartsubstruct
    
    \scnheader{Предметная область языка логики высказываний}
    \scniselement{предметная область}
    \scnsdmainclasssingle{Язык логики высказываний}
    %\scnsdclass{Язык логики высказываний}
    
    \scnheader{Язык логики высказываний}
    \scniselement{логический язык}
    \scnsubset{язык представления методов}
    \scndefinition{формальный язык, предназначенный для анализа логической структуры сложных высказываний.}
    
    \bigskip
    \scnendstruct
    
    \scnstartsubstruct
    
    \scnheader{Предметная область языка логики предикатов}
    \scniselement{предметная область}
    \scnsdmainclasssingle{Язык логики предикатов}
    %\scnsdclass{Язык логики предикатов}
    
    \scnheader{Язык логики предикатов}
    \scniselement{логический язык}
    \scnsubset{язык представления методов}
    \scndefinition{формальный язык, предназначенный для анализа логической структуры простых высказываний.}
    
    \bigskip
    \scnendstruct
    
    \scnstartsubstruct
    
    \scnheader{Предметная область языка логического вывода}
    \scniselement{предметная область}
    \scnsdmainclasssingle{логический вывод}
    \scnsdclass{правило вывода;процесс логического вывода;аксиомная схема;предпосылка;заключение}
    \scnsdrelation{выводимость*;правила вывода*;равносильные преобразования*;аксиомные схемы логики*;применение аксиомной схемы*}
    
    \scnheader{предпосылка}
    \scndefinition{исходное суждение логического вывода}
    
    \scnheader{заключение}
    \scndefinition{новое суждение логического вывода}
    
    \scnheader{логический вывод}
    \scndefinition{рассуждение, которое описывает переход от предпосылок к заключениям.}
    
    \scnheader{процесс логического вывода}
    \scnsubset{процесс}
    \scndefinition{процесс рассуждения, в ходе которого осуществляется переход от некоторых предпосылок к заключениям.}
    
    \scnheader{выводимость*}
    \scndefinition{отношение, существующее между предпосылками и заключением рассуждения.}
    \scnexplanation{Над формулами исчисления с помощью правил вывода задается отношение выводимости.}
    
    \scnheader{правило вывода}
    \scnidtf{правило преобразования некоторой формальной системы}
    \scndefinition{допустимые способы переходов от некоторой совокупности утверждений, называемых посылками, к некоторому определённому утверждению — заключению.}
    
    \scnheader{правила вывода*}
    \scnidtf{быть правилами вывода логики*} 
    \scnidtf{быть правилами вывода логической модели*}
    \scndefinition{отношение, существующее между логикой и правилом вывода.}
    
    \scnheader{равносильные преобразования*}
    \scndefinition{отношение, существующее между логикой и преобразованием формул логики, которые принимают одинаковые логические значения при любом наборе значений входящих в формулы элементарных высказываний.}
    
    \scnheader{аксиомная схема}
    \scndefinition{формула, верная без доказательства, переменные которой понимаются как произвольные формулы.}
    
    \scnheader{аксиомные схемы логики*}
    \scndefinition{отношение, существующее между логикой и аксиомной схемой.}
    
    \scnheader{применение аксиомной схемы*}
    \scndefinition{отношение, существующее между формулой  с аксиомной схемой и результатом применения аксимной схемы.}
    
    \scnheader{следует отличать*}
    \scnhaselementset{логический вывод
    	;процесс логического вывода
    	;выводимость*
    }
    \scnhaselementset{правило вывода
    	;правила вывода*
    }
    \scnhaselementset{правила вывода*
    	;равносильные преобразования*
    }
    
    \scnheader{Примеры логических выводов}
    \scneqtoset{\scgfileitem{figures/sd_logical_languages/example_logical_conclusion_1.png}\\
    	\scnaddlevel{1}
    	\scnrelfrom{описание примера}{Пример логического вывода в логике высказываний, который является доказательством формулы (A$\rightarrow$A).}
    	\scnaddlevel{-1}
    	;
    	\scgfileitem{figures/sd_logical_languages/example_logical_conclusion_2.png}\\
    	\scnaddlevel{1}
    	\scnrelfrom{описание примера}{Пример логического вывода в логике высказываний, который является доказательством формулы (A $\vdash$ (A$\rightarrow$B).}
    	\scnaddlevel{-1}
    }
    
    \bigskip
    \scnendstruct
    
    \scnstartsubstruct
    
    \scnheader{Предметная область логических моделей решения задач}
    \scniselement{предметная область}
    \scnsdmainclasssingle{логическая модель решения задач}
    \scnsdclass{логический метод решения задач;предикат;логика высказываний;логика предикатов;нечеткая логика;темпоральная логика}
    
    \scnheader{логическая модель решения задач}
    \scnrelto{включение}{метод решения задач}
    \scnidtf{логика}
    \scnidtf{метаметод интерпретации соответствующего класса логических методов}
    
    \scnheader{логический метод решения задач}
    \scnrelto{включение}{метод решения задач}
    \scnidtf{логический вывод}
    \scnexplanation{\textit{вид знаний}, хранимых в \textit{памяти кибернетической системы} и содержащих информацию, которой достаточно либо для сведения каждой \textit{задачи} из соответствующего \textit{класса логических задач} к \textit{полной системе подзадач*}, решение которых гарантирует решение исходной \textit{задачи}, \uline{либо} для окончательного решения этой \textit{задачи} из указанного \textit{класса логических задач}}
    
    \scnheader{логика высказываний}
    \scniselement{логическая модель решения задач}
    \scnidtf{раздел символической логики, изучающий сложные высказывания, образованные из простых, и их взаимоотношения.}
    \scnexplanation{Правила вывода: Modus ponens и правило обобщения, правила введения и удаления для пропозициональных связок, правила противоречия и сведения к противоречию.}
    \scnrelfromset{правила вывода}{Modus ponens*
    	;правило монотонности*
    }
    \scnrelfrom{равносильные преобразования}{Множество равносильных преобразований логики высказываний}
    \scnrelfrom{логический язык}{Язык логики высказываний}
    \scnrelfrom{аксиомные схемы логики}{Множество аксиомных схем логики высказываний}
    
    \scnheader{Множество равносильных преобразований логики высказываний}
    \scnhaselement{ассоциативность*}
    \scnhaselement{коммутативность*}
    \scnhaselement{дистрибутивность*}
    \scnhaselement{идемпотентность*}
    \scnhaselement{двойное отрицание*}
    \scnhaselement{правило де Моргана*}
    \scnhaselement{свойство констант*}
    \scnhaselement{закон противоречия*}
    \scnhaselement{закон исключения третьего*}
    \scnhaselement{выражение связок*}
    \scnhaselement{поглощение*}
    \scnhaselement{склеивание*}
    \scnhaselement{правило перевертывания*}
    
    \scnheader{Множество аксиомных схем логики высказываний}
    \scnhaselement{аксиомня схема введения импликации}
    \scnhaselement{аксиомня схема введения конъюнкции}
    \scnhaselement{аксиомня схема введения дизъюнкции}
    \scnhaselement{аксиомня схема введения отрицания}
    \scnhaselement{аксиомня схема введения эквивалентности}
    \scnhaselement{аксиомня схема удаления импликации}
    \scnhaselement{аксиомня схема удаления конъюнкции}
    \scnhaselement{аксиомня схема удаления дизъюнкции}
    \scnhaselement{аксиомня схема удаления отрицания}
    \scnhaselement{аксиомня схема удаления эквивалентности}
    \scnhaselement{аксиомня схема введения эквивалентности}
    \scnhaselement{аксиомня схема удаления импликации}
    
    \scnheader{логика предикатов}
    \scniselement{логическая модель решения задач}
    \scnidtf{раздел символической логики, изучающий рассуждения и другие языковые контексты с учётом внутренней структуры входящих в них простых высказываний, при этом выражения языка трактуются функционально, то есть как знаки некоторых функций или же как знаки аргументов этих функций.}
    \scnrelfromset{правила вывода}{Modus ponens*
    	;правило удаления существования*
    	;правило введения всеобщности*
    }
    \scnrelfrom{равносильные преобразования}{Множество равносильных преобразований логики предикатов}
    \scnrelfrom{логический язык}{Язык логики предикатов}
    \scnrelfrom{аксиомные схемы логики}{Множество аксиомных схем логики предикатов}
    
    \scnheader{Множество равносильных преобразований логики предикатов}
    \scnrelfrom{включение}{Множество равносильных преобразований логики высказываний}
    \scnhaselement{коммутативность кванторов*}
    \scnhaselement{дистрибутивность кванторов*}
    \scnhaselement{вынос констант*}
    \scnhaselement{двойственность кванторов*}
    
    \scnheader{Множество аксиомных схем логики предикатов}
    \scnrelfrom{включение}{Множество аксиомных схем логики высказываний}
    \scnhaselement{аксиомня схема введения существования}
    \scnhaselement{аксиомня схема удаления всеобщности}
    
    \scnheader{нечеткая логика}
    \scniselement{логическая модель решения задач}
    \scnidtf{раздел многозначной логики, который базируется на обобщении классической логики и теории нечётких множеств для формализации нечётких знаний, характеризуемых лингвистической неопределённостью.}
    % \scnrelfromset{правила вывода}{Modus ponens*;правило удаления существования*;правило введения всеобщности*}
    \scnrelfrom{равносильные преобразования}{Множество равносильных преобразований нечеткой логики}
    \scnrelfrom{логический язык}{Язык нечеткой логики}
    \scnrelfrom{аксиомные схемы логики}{Множество аксиомных схем нечеткой логики}
    
    \scnheader{нечеткое множество}
    \scnexplanation{\textbf{\textit{нечеткое множество}} – это \textit{множество}, которое представляет собой совокупность элементов произвольной природы, относительно которых нельзя точно утверждать – обладают ли эти элементы некоторым характеристическим свойством, которое используется для задания этого нечеткого множества. Принадлежность элементов такому множеству указывается при помощи \textit{нечетких позитивных sc-дуг принадлежности}.}
    
    \scnheader{темпоральная логика}
    \scniselement{логическая модель решения задач}
    \scnidtf{раздел неклассической логики, в рамках которого изучаются свойства высказываний с истинностными значениями, изменяющимися во времени.}
    % \scnrelfromset{правила вывода}{Modus ponens*;правило удаления существования*;правило введения всеобщности*}
    \scnrelfrom{равносильные преобразования}{Множество равносильных преобразований темпоральной логики}
    \scnrelfrom{логический язык}{Язык темпоральной логики}
    \scnrelfrom{аксиомные схемы логики}{Множество аксиомных схем темпоральной логики}
    
    \bigskip
    \scnendstruct
    
    \scnendcurrentsectioncomment
    
    \end{SCn}

\scsubsection[\scneditors{Садовский М.Е.;Никифоров С.А.}\protect\scnmonographychapter{Глава 2.6. Языковые средства формального описания синтаксиса и денотационной семантики различных языков в интеллектуальных компьютерных системах нового поколения}\protect\scnmonographychapter{Глава 4.1. Структура интерфейсов интеллектуальных компьютерных систем нового поколения}]{Предметная область и онтология файлов, внешних информационных конструкций и внешних языков ostis-систем}
\label{sd_files}
\input{Contents/part_kb/sd_files.tex}

\scsubsubsection[\scneditors{Никифоров С.А.;Бобёр  Е.С.}\protect\scnmonographychapter{Глава 2.6. Языковые средства формального описания синтаксиса и денотационной семантики различных языков в интеллектуальных компьютерных системах нового поколения}]{Предметная область и онтология естественных языков}
\label{sd_natural_languages}
\input{Contents/part_kb/sd_natural_languages.tex}

\scparagraph[\scneditors{Никифоров С.А.;Бобёр  Е.С.}\protect\scnmonographychapter{Глава 2.6. Языковые средства формального описания синтаксиса и денотационной семантики различных языков в интеллектуальных компьютерных системах нового поколения}]{Предметная область и онтология синтаксиса естественных языков}
\label{sd_syntax_natural_lang}

\scparagraph[\scneditors{Никифоров С.А.;Бобёр  Е.С.}\protect\scnmonographychapter{Глава 2.6. Языковые средства формального описания синтаксиса и денотационной семантики различных языков в интеллектуальных компьютерных системах нового поколения}]{Предметная область и онтология денотационной семантики естественных языков}
\label{sd_sem_natural_lang}

\scsubsection[\scneditors{Никифоров С.А.;Шункевич Д.В.}\protect\scnmonographychapter{Глава 3.1. Формализация понятий действия, задачи, метода, средства, навыка и технологии}]{Глобальная предметная область и онтология, описывающая воздействия, действия, методы, средства и технологии}
\label{sd_actions}
\input{Contents/part_kb/sd_actions.tex}

\scsubsubsection[\scnmonographychapter{Глава 3.1. Формализация понятий действия, задачи, метода, средства, навыка и технологии}]{Предметная область и онтология локальных предметных областей и онтологий действий}
\label{local_sd_actions}

\scsubsubsection[\scnidtf{Типология неавтоматизированных ("вручную"{} выполняемых) и автоматически выполняемых \textit{действий}, направленных на управление процессами выполнения различных \textit{сложных действий}, а также система понятий, используемая для \textit{управления сложными действиями}}]{Предметная область и онтология действий по управлению деятельностью многоагентных систем}
\label{local_sd_project_management}